\section*{ВВЕДЕНИЕ}
\addcontentsline{toc}{section}{ВВЕДЕНИЕ}

В современном мире, где цифровые технологии проникают во все сферы жизни, веб-приложения становятся неотъемлемой частью повседневного общения и взаимодействия людей. Одним из наиболее востребованных типов веб-приложений является чат, позволяющий пользователям обмениваться сообщениями в реальном времени. Такие приложения находят широкое применение как в личной, так и в профессиональной сфере, обеспечивая оперативное общение, координацию и совместную работу.

Разработка веб-приложения чата представляет собой сложный и многогранный процесс, включающий в себя выбор подходящих технологий, проектирование архитектуры системы, реализацию функциональных компонентов и обеспечение безопасности данных. Успешное создание такого приложения требует не только глубоких знаний и навыков в области веб-разработки, но и понимания особенностей и требований пользователей, а также современных трендов в сфере информационных технологий.

\emph{Цель данной работы} заключается в создании веб-приложения, обеспечивающего возможность пользователей обмениваться сообщениями в чат-комнатах. Для достижения этой цели необходимо разработать серверную часть приложения, предоставляющую возможность независимо от нее разрабатывать различные клиенты для разных платформ. Для этого необходимо решить следующие задачи:

\begin{itemize}
	\item разработать требования к программной системе; определить модель данных и варианты использования;
	\item спроектировать архитектуру программной системы; определить маршруты приложения; произвести выбор технологий для программной реализации;
	\item реализовать модули программной системы с использованием выбранных технологий; реализовать пользовательский интерфейс в виде веб-сайта для демонстрации работы приложения; провести системное тестирование программной системы. 
\end{itemize}

\emph{Структура и объем работы.} Отчет состоит из введения, 4 разделов основной части, заключения, списка использованных источников, 2 приложений. Текст выпускной квалификационной работы равен 153 страницам.

\emph{Во введении} сформулирована цель работы, поставлены задачи разра-
ботки, описана структура работы, приведено краткое содержание каждого из разделов.

\emph{В первом разделе} приводится анализ предметной области, рассмотрено понятие чата, роль чатов в обществе, их основные виды и история развития. 

\emph{Во втором разделе} на определяются требования к разрабатываемой программной системе.

\emph{В третьем разделе} на стадии технического проектирования представленf архитектура программнойсистемы.

\emph{Четвертый этап} представляет собой рабочий проект. В этом разделе описаны классы приложения, их поля и методы. Также здесь приведены результаты системного тестирования программной системы.

\emph{В заключении} описаны результаты проведенной работы.

В приложении А представлен графический материал.
В приложении Б содержатся фрагменты исходного кода созданной
программно-информационной системы