\addcontentsline{toc}{section}{СПИСОК ИСПОЛЬЗОВАННЫХ ИСТОЧНИКОВ}

\begin{thebibliography}{9}
	\bibitem{aspnetexamples} Фримен, А. ASP.NET Core 3 с примерами на C\# для профессионалов / А. Фримен. – Москва: Вильямс, 2021. – 1184 с. – 978-5-907365-46-9. – Текст~: непосредственный.
	\bibitem{cssecurity} Венц, К. Безопасность ASP. NET Core / М. Прайс. – Москва : ДМК Пресс, 2023. – 386 с. – ISBN 978-5-93700-176-4. – Текст~: непосредственный.
	\bibitem{csharp} Прайс, М. C\# 10 и .NET 6. Современная кросс-платформенная разработка / М. Прайс. – Санкт-Петербург : Питер, 2023. – 848 с. – ISBN 978-5-4461-2249-3. – Текст~: непосредственный.
	\bibitem{csdocs} Руководство по C\# | learn.microsoft.com : сайт. – URL: https://learn.microsoft.com/ru-ru/dotnet/csharp/ (дата обращения: 24.04.2024).
	\bibitem{aspnetdocs} Документация по ASP.NET | learn.microsoft.com : сайт. – URL: https://learn.microsoft.com/ru-ru/aspnet/core/?view=aspnetcore-8.0 (дата обращения: 26.04.2024).
	\bibitem{ef} Центр документации Entity Framework | learn.microsoft.com : сайт. – URL: https://learn.microsoft.com/ru-ru/ef/ (дата обращения: 25.04.2024).
	\bibitem{db} Осипов, Д.Л. Технологии проектирования баз данных / Д.Л. Осипов. – Москва : ДМК Пресс, 2019. – 498 с. – ISBN 978-5-97060-737-4. – Текст~: непосредственный.
	\bibitem{postgres} Левшин, И.В. Postgres. Первое знакомство / И.В. Левшин, П.В. Лузанов, Е.В. Рогов. – Москва : ДМК Пресс, 2023. – 178 с. – ISBN 978-5-6045970-1-9. – Текст~: непосредственный.
	\bibitem{sql} Грофф Д. Р. SQL. Полное руководство / Д. Р. Грофф, П. Н. Вайнберг, Э. Д. Опель. – 3-е изд. – Москва : Диалектика-Вильямс, 2020. – 960 с. - ISBN - 978-5-907114-26-5. Текст: непосредственный.
	\bibitem{webapi} Лоре, А. Проектирование веб-API / А. Лоре. – Москва : ДМК Пресс, 2020. – 440 с. – ISBN ISBN 978-5-97060-861-6. – Текст~: непосредственный.
	\bibitem{http} Поллард, Б. HTTP/2 в действии / Б. Поллард. – Москва~: ДМК Пресс, 2021. – 424 с. – ISBN 978-5-97060-925-5. – Текст~: непосредственный.
	\bibitem{refactor} Рефакторинг и Паттерны проектирования | refactoring.guru : сайт. – URL: https://refactoring.guru/ru (дата обращения: 08.05.2024).
	\bibitem{uml} Буч, Г. Введение в UML от создателей языка / Г. Буч, И. Якобсон, Д. Рамбо. – Москва : ДМК Пресс, 2015. – 498 с. – ISBN 978-5-457-43379-3. – Текст : непосредственный.
	\bibitem{umlproj} Флегонтов, А. В. Моделирование информационных систем. Unified Modeling Language / А.В. Флегонтов, И.Ю. Матюшичев. – Санкт-Петербург : Лань, 2023. – 140 с. – ISBN 978-5-8114-4274-4. – Текст : непосредственный.
	\bibitem{nginx} де Йонге, Д. NGINX. Книга рецептов / Д. де Йонге. – Москва~: ДМК Пресс, 2019. – 176 с. – ISBN 978-5-97060-790-9. – Текст~: непосредственный.
	\bibitem{js} Флэнаган, Д. JavaScript. Полное руководство / Д. Флэнаган. – Москва~: Диалектика-Вильямс, 2021. – 720 с. – ISBN 978-5-907203-79-2. – Текст~: непосредственный.
	\bibitem{react} Стефанов, С. React. Быстрый старт / С. Стефанов. – Санкт-Петербург : Питер, 2023. – 304 с. – ISBN 978-5-4461-2115-1. – Текст~: непосредственный.
	\bibitem{nextjs} Docs | nextjs.org : сайт. – URL: https://nextjs.org/docs (дата обращения: 06.05.2024).
	\bibitem{htmlcss} Дакетт, Дж. HTML и CSS. Разработка и дизайн веб-сайтов / Дж. Дакетт. – Москва : Эксмо, 2020 – 480 с. – ISBN 978-5-04-101286-1. – Текст~: непосредственный.
	\bibitem{webtech} Web technology for developers | developer.mozilla.org : сайт. – URL: https://developer.mozilla.org/en-US/docs/Web (дата обращения: 07.05.2024).
\end{thebibliography}