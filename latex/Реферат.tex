\abstract{РЕФЕРАТ}

Объем работы равен \formbytotal{lastpage}{страниц}{е}{ам}{ам}. Работа содержит \formbytotal{figurecnt}{иллюстраци}{ю}{и}{й}, \formbytotal{tablecnt}{таблиц}{у}{ы}{}, \arabic{bibcount} библиографических источников и \formbytotal{числоПлакатов}{лист}{}{а}{ов} графического материала. Количество приложений – 2. Графический материал представлен в приложении А. Фрагменты исходного кода представлены в приложении Б.

Перечень ключевых слов: веб-приложение, чат-приложение, ASP.NET Core, Next.js, серверная часть, клиентская часть, веб-разработка, JavaScript, React, Универсальное приложение, серверный рендеринг, статическая генерация страниц, динамическое обновление, маршрутизация, REST API, авторизация и аутентификация, пользовательский интерфейс, фронтенд, бэкенд, компонентная архитектура, взаимодействие клиента и сервера, безопасность данных, масштабируемость, производительность, тестирование.

Объектом разработки является веб-приложение, обеспечивающее коммуникацию пользователей приложения посредством обмена сообщениями внури чат-комнат.

Целью выпускной квалификационной работы является обеспечение заинтересованных пользователей сервисом, который будет позволять им общаться через глобальную сеть Интернет.

В ходе разработки приложения были выявлены ключевые элементы, которые были организованы в виде информационных блоков. Для эффективной работы с этими элементами были использованы классы и методы соответствующих модулей, обеспечивая связь с сущностями предметной области. Кроме того, были спроектированы и внедрены разделы, необходимые для обеспечения внутренней коммуникации между пользователями в рамках приложения, гарантируя тем самым корректное функционирование веб-сайта.


\selectlanguage{english}
\abstract{ABSTRACT}
  
The volume of work is \formbytotal{lastpage}{page}{}{s}{s}. The work contains \formbytotal{figurecnt}{illustration}{}{s}{s}, \formbytotal{tablecnt}{table}{}{s}{s}, \arabic{bibcount} bibliographic sources and \formbytotal{числоПлакатов}{sheet}{}{s}{s} of graphic material. The number of applications is 2. The graphic material is presented in annex A. The layout of the site, including the connection of components, is presented in annex B.

List of key words: web application, chat application, ASP.NET Core, Next.js, server-side, client-side, web development, JavaScript, React, Universal application, server-side rendering, static page generation, dynamic updating, routing, REST API, authentication and authorization, user interface, frontend, backend, component-based architecture, client-server interaction, data security, scalability, performance, testing.

The object of development is a web application facilitating user communication through message exchange within chat rooms.

The aim of the final qualification work is to provide interested users with a service that will allow them to communicate via the global Internet.

During the application development process, key elements were identified, organized as informational blocks. Classes and methods of corresponding modules were utilized to effectively work with these elements, ensuring connectivity with domain entities. Additionally, sections necessary for internal communication between users within the application were designed and implemented, thereby guaranteeing the correct functioning of the website.
\selectlanguage{russian}
